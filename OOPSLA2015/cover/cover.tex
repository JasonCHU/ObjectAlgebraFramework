%%%%%%%%%%%%%%%%%%%%%%%%%%%%%%%%%%%%%%%%%
% Frequently Asked Questions
% LaTeX Template
% Version 1.0 (22/7/13)
%
% This template has been downloaded from:
% http://www.LaTeXTemplates.com
%
% Original author:
% Adam Glesser (adamglesser@gmail.com)
%
% License:
% CC BY-NC-SA 3.0 (http://creativecommons.org/licenses/by-nc-sa/3.0/)
%
%%%%%%%%%%%%%%%%%%%%%%%%%%%%%%%%%%%%%%%%%

\documentclass[11pt]{article}

\usepackage[margin=1in]{geometry} % Required to make the margins smaller to fit more content on each page
\usepackage[linkcolor=blue]{hyperref} % Required to create hyperlinks to questions from elsewhere in the document
\hypersetup{pdfborder={0 0 0}, colorlinks=true, urlcolor=blue} % Specify a color for hyperlinks
\usepackage{todonotes} % Required for the boxes that questions appear in
\usepackage{tocloft} % Required to give customize the table of contents to display questions
\usepackage{microtype} % Slightly tweak font spacing for aesthetics
\usepackage{palatino} % Use the Palatino font
\usepackage{listings}

\setlength\parindent{0pt} % Removes all indentation from paragraphs

% Create and define the list of questions
\newlistof{questions}{faq}{\large List of Frequently Asked Questions} % This creates a new table of contents-like environment that will output a file with extension .faq
\setlength\cftbeforefaqtitleskip{4em} % Adjusts the vertical space between the title and subtitle
\setlength\cftafterfaqtitleskip{1em} % Adjusts the vertical space between the subtitle and the first question
\setlength\cftparskip{.3em} % Adjusts the vertical space between questions in the list of questions

% Create the command used for questions
\newcommand{\question}[1] % This is what you will use to create a new question
{
\refstepcounter{questions} % Increases the questions counter, this can be referenced anywhere with \thequestions
\par\noindent % Creates a new unindented paragraph
\phantomsection % Needed for hyperref compatibility with the \addcontensline command
\addcontentsline{faq}{questions}{#1} % Adds the question to the list of questions
\todo[inline, color=white!40]{\textbf{#1}} % Uses the todonotes package to create a fancy box to put the question
\vspace{1em} % White space after the question before the start of the answer
}

\lstset{ %
language=Java,                % choose the language of the code
columns=flexible,
lineskip=-1pt,
basicstyle=\ttfamily\small,       % the size of the fonts that are used for the code
numbers=none,                   % where to put the line-numbers
numberstyle=\ttfamily\tiny,      % the size of the fonts that are used for the line-numbers
stepnumber=1,                   % the step between two line-numbers. If it's 1 each line will be numbered
numbersep=5pt,                  % how far the line-numbers are from the code
backgroundcolor=\color{white},  % choose the background color. You must add \usepackage{color}
showspaces=false,               % show spaces adding particular underscores
showstringspaces=false,         % underline spaces within strings
showtabs=false,                 % show tabs within strings adding particular underscores
%  frame=single,                   % adds a frame around the code
tabsize=2,                  % sets default tabsize to 2 spaces
captionpos=none,                   % sets the caption-position to bottom
breaklines=true,                % sets automatic line breaking
breakatwhitespace=false,        % sets if automatic breaks should only happen at whitespace
title=\lstname,                 % show the filename of files included with \lstinputlisting; also try caption instead of title
escapeinside={(*}{*)},          % if you want to add a comment within your code
keywordstyle=\ttfamily\bfseries,
deletekeywords={label}
% commentstyle=\color{Gray},
% stringstyle=\color{Green}
}

\newcommand{\hl}[1]{\textcolor{red}{#1}}

% Uncomment the line below to get rid of the trailing dots in the table of contents
%\renewcommand{\cftdot}{}

% Uncomment the two lines below to get rid of the numbers in the table of contents
%\let\Contentsline\contentsline
%\renewcommand\contentsline[3]{\Contentsline{#1}{#2}{}}

\begin{document}

%----------------------------------------------------------------------------------------
%	TITLE AND LIST OF QUESTIONS
%----------------------------------------------------------------------------------------

\begin{center}
\Huge{Cover Letter} % Main title
\end{center}

%\listofquestions % This prints the subtitle and a list of all of your questions

%\newpage % Comment this if you would like your questions and answers to start immediately after table of questions

%----------------------------------------------------------------------------------------
%	QUESTIONS AND ANSWERS
%----------------------------------------------------------------------------------------

\vspace{.5in}

This cover letter describes the changes made in paper \#60 ``Scrap your Boilerplate with Object Algebras'' for OOPSLA '15 final submission, regarding the reviewers' comments in stage 1 revision.\\

Besides the specific issues listed below, all the minor ones (typos, grammar and figure positioning) have been fixed.

\newpage

\section*{OOPSLA '15 Review \#60A}

\question{The paper is well presented, except for the hyphenation, which is horrible in places.}

We have addressed the issues with hyphenation now.

\question{It might be interested to compare this with other Aspect-Oriented
approaches to traversals as well as AOOD, e.g. Hannemann \& Kiczales
OOPSLA 2002 (I don't know if it is, because I'm on a plane without the
Interwebz) or others from AOP-land. }

This is done now, there is a paragraph called ``Eliminating Boilerplate
in Design Patterns'' in related work.

\question{p7 should \lstinline{StatAlg} extend \lstinline{ExpAlg}?  if not, explain.}

We added some text explaining the option not to have \lstinline{StatAlg} extending \lstinline{ExpAlg}.

\question{p11 where did the QL programs come from?}

The QL programs represent questionnaires describing binary search
problem and were generated.  We now included this in the text.

\newpage

\section*{OOPSLA '15 Review \#60B}

\question{What is the code size difference between the Vanilla and Shy solutions?
Please give a rough estimate in the author response (with/without boilerplate
that could be generated).}

\hl{TODO: Tijs}

\question{What technique was used for generating the boilerplate interfaces? Is the
annotation processor in javax used, or what?}

We added a new (small) section (Section 10) about the implementation
of Shy. This section explains what annotation processor has been used.

\question{Why do you have so many outliers in the diagrams, in particular for fig 19.
Did you not take averages? What are the confidence intervals? Did you measure
start-up or warmed-up performance?}

We didn't measure warmup time: the benchmarks were run first to
warm-up; then to measure warmup-up performance. The initial outliers
were caused by garbage collection pauses. The current measurements
were run with 2GB of memory to avoid GC pauses.

\begin{verbatim}
$ java -Xmx2048M -Xms512M  -server -d64 -cp bin:lib/Library.jar:lib/ant
lr-4.2.2-complete.jar:../naked-object-algebras/bin _ast.benchmark.Bench
mark
\end{verbatim}

\newpage

\section*{OOPSLA '15 Review \#60C}

\question{Are top-down traversals supported? How are the
desugaring transformations implemented?}

We added a new section (Section 8) which discusses how desugaring
transformations are implemented. In that section we also discuss some
limitations, including mentioning that top-down traversals are not supported.

\question{I got the impression that the paper claims compositionality. However, that
may be a wrong impression. The extensibility that the paper does claim is
indeed supported. Unfortunately it is asymmetric due to single inheritance
in Java. Would Scala traits make the approach fully compositional?}

Regarding extensibility, we can currently get quite far with Java 8
using interfaces with default methods. Using default methods allows us
to get quite a few of of the benefits of multiple-inheritance and
traits. Of course if we had traits we could do some things more
elegantly. We hope that the discussion about ``Independent
Extensibility'' in Section 9.2 illustrate the benefits of using
interfaces with default methods.

\question{There is no explanation of the implementation of the generator or its
properties. Is it based on byte-code transformation? Does  it actually exist
or is it conceptual? I would have expected a bit more explanation about the
interaction with the IDE; where does the generated code live?}

Please see the new Section on implementation (Section 10), which
provides answers to all of the questions above.

\question{Origins of strategic programming in related work is mis-represented.}

We have changed the formulation in the text to accurately reflect the origins of strategic programming. 


\question{p5: ��the code in Figure 4 is much shorter�� You write in multiple places
that the code is so short, which seems to refer to its textual size. All
the code fragments that are shown are pretty short. It would be more informative
if the conciseness would be analytically characterized in terms of number of
method definitions or perhaps more fine-grained, in terms of statements or
method calls.}

In the case-study we (Table 1) we measure the number of factory methods ("expression constructors") that had to be overridden relative to the total number of case.


\question{p5/fig5: that rename is rather lame; why not a proper rename with a new name?}

This has been done now: the renaming example in Section 2 is now a
proper rename.

\question{p8: ��Since Object Algebras are factories, the transformation is executed
immediately during construction of tree structures.�� It would be useful at
this point to show how such a transformation is invoked and how a transformation
pipeline is constructed.}

Section 6.1 now includes an example of how to invoke the SubsVar transformation.
This transformation transforms into the FreeVars algebra. 


\question{p8: ``To execute substitution, SubstVar should be subclassed with implementations
for \lstinline{x()}, \lstinline{e()}, and \lstinline{expAlg()}.'' That is a cumbersome way to construct closures;
sounds rather boilerplaty; why not show such an invocation.}

See client code in Section 6.1.

\question{p9/fig13: That produces a fold/bottom-up transformation. How about top-down
traversals? Later in the paper you mention a desugaring, but the code for the
desugaring is never shown. Typically, desugarings are defined using top-down
traversals, often pattern matching on combinations of constructor patterns
(deep pattern matching). Is that supported in this approach?}

Please see Section 8, where we now discuss how to do desugaring
transformations and limitations of the approach.

\question{For that matter, the transformation algebras represent type preserving
transformations; how are translations (between types) expressed?}

\hl{(SHOULD WE INCLUDE THIS TEXT?)
Queries are actually an example of a ``transformation'' that changes the
structure. It transforms from an AST into a monoid structure. The fact
that we need to build a monoid structure as the result is expressed by
using \lstinline{m()} as a factory.}

\question{p10: The composition of these transformations is neat! Unfortunately, it
seems to be assymetric, if I understand things well. One would really like
to have independent implementations of an algebra (syntax) and implementation of
a transformation and then compose thge implementations. As I understand the code,
the implementation of \lstinline{UniqueWithLambda<E>} inherits the implementation of \lstinline{Unique<E>},
but cannot inherit the implementation of a \lstinline{Unique} for \lstinline{LamAlgTransform}, since that
would require multiple inheritance. Is that a correct analysis? And then it seems
an inherent problem for true compositionally in Java. Would Scala traits solve
this issue?}

Added a subsection about independent extensibility to show symmetric extension of the transformation.
Java 8 interfaces with default methods provide almost the same expressivity as Scala traits.

\question{p10: I��m quite curious about the implementation of desugaring; why was it not
included (in the Appendix)?}

Added section on compositional desugarings (Section 8).

\question{p11: Can the desugar transformations be combined into a single traversal? How about deforestation?}

The transformations are always executed in a single bottom-up
traversal. This is explained in the new section on
desugaring. Non-desugared trees are never built, the desugarings are
automatically deforested.

\question{p12: ``Strategic programming is an approach to data structure traversal, which
originated in term rewriting [3, 24, 25].'' This sentence is not historically
accurate. Generic traversals in term rewriting were first introduced in
[26] (Stratego), which extended the strategies of Elan [3] with 'structure-shy'
generic one-level tree traversal operators (all, one, some), which could be
combined with strategies in a wide range of generic traversals. [25] describes
a core language for rewriting based on that approach. ``Scrap your boilerplate''
and earlier work of Laemmel et al. are implementations of Stratego generic
traversals in Haskell and Java. The extension of ASF with traversals [24]
came five years later and post-dates most of the work that is described in
the next paragraphs. Don��t cite is as the origin of strategic programming!}

This has been fixed in the text. 


\end{document}
