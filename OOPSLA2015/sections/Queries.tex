\section{Queries}\label{sec:queries}

This section shows the ideas behind generic queries and how they are
implemented in \Name.  A query is an operation that traverses a
structure and computes some aggregate value. The inspiration for
queries comes from similar types of traversals used in functional
programming libraries, such as ``Scrap your
Boilerplate''~\cite{ralf03syb}.

%We motivate queries by walking through a number
%of different ways in which such operations are implemented.

Fig.~\ref{exp_alg} shows a simple type of expressions, represented
as the object algebra interface \lstinline{ExpAlg}. We will use this
minimal object algebra interface throughout the rest of the paper to
illustrate the various different types of traversals supported by
\Name.  Three different kinds of nodes exist: a numeric literal, a
variable or the addition of two expressions. Queries are illustrated
by implementing an operation to compute the free variables in an expression.

\begin{comment}
As a specific type of object algebras, queries allow users to define
new operations handling a user-defined data structure\bruno{This
  definition of queries is just too broad. Please
look at papers like Syb to see how they describe queries and transformations}. A \textit{query
  algebra} is a class implementing an object algebra interface by a
top-down traversal throughout the hierarchy. It is something
supporting the program to gather information from the substructures of
a data type recursively, and make a response at the root node to the
query.
\end{comment}

\subsection{Boilerplate Queries}\label{subsec:freevars}

\begin{figure}[t]
\nocaptionrule
\lstinputlisting[linerange=6-11]{../ObjectAlgebras/src/expDemo3/ExpAlg.java} % APPLY:linerange=EXP_ALG
\caption{The \lstinline{ExpAlg} object algebra interface.}
\label{exp_alg}
\end{figure}

Fig.~\ref{freevars_without_monoid} shows a
standard approach for computing free variables using object algebras\footnote{Here and in the following we will use interfaces with \lstinline{default}-methods (as introduced in Java 8) to combine queries and transformations using multiple inheritance.}.
A set of strings is used to store the names of the free variables. The
\lstinline{Var} method returns a singleton set of \lstinline{s},
whereas the \lstinline{Lit} method returns an empty set. The
more interesting case is in the \lstinline{Add} method, where the two
sets are joined into one.

\begin{figure}[t]
\nocaptionrule
\lstinputlisting[linerange=12-18]{../ObjectAlgebras/src/expDemo2/FreeVars.java} % APPLY:linerange=FREEVARS_WITHOUT_MONOID
\caption{Free variables as an object algebra.}
\label{freevars_without_monoid}
\end{figure}
\begin{comment}
\bruno{put code in a figure, add a caption and refer
  to the figure in the text.}
\end{comment}

The typical pattern of a query is to collect some information from
some of the nodes of the structure, and to aggregate the information
that comes from multiple child nodes. For example, in the case of free
variables, the strings from the \lstinline{Var} nodes are collected,
and in the \lstinline{Add} nodes the information from multiple
children is merged into a single set. An important observation about
queries is that the code to aggregate information tends to be the
same: if we had a subtraction node, the code would be essentially
identical to \lstinline{Add}. Moreover, there are only very few types
of nodes that contain relevant information for the query. For nodes
that contain information that is not relevant to the query, we simply
return the a neutral value (such as the empty set in \lstinline{Lit}).
Nonetheless, a programmer has to write this boring boilerplate code
handling the traversals.  While for the small structure presented here
this may not look too daunting, in a large structure with dozens or
even hundreds of constructors such code becomes a significant burden.

%Moreover, it is a query-based
%approach: you still have to write a bunch of similar stuff with a
%different query raised, for instance, a pretty printer.

\subsection{Abstracting Generic Traversal Code}\label{subsec:genericquery}

Clearly a better approach would be to abstract the generic traversal
code for queries. This way, when programmers need to implement query
operations, they can simply reuse the generic traversal code and
focus only on dealing with the nodes that do something interesting.

%Queries are so similar actually: a user has to indicate the rules in
%which the program may address cases on primitive types and ``append''
%the information. With these two issues, everything becomes simple in
%the traversal. Hence we introduce the concept of monoid and generic
%traversal here in our query algebras.

%% \begin{figure}[tb]
%% \lstinputlisting[linerange=5-8]{../ObjectAlgebras/src/expDemo3/Monoid.java} % APPLY:linerange=MONOID
%% \vspace{-.1in}
%% \caption{The monoid interface.}
%% \label{monoid}
%% \end{figure}

The code that captures the aggregation and collection of information
can be captured by a well-known algebraic structure called a
\emph{monoid}. Monoids are commonly used in functional programming
for such purposes, but they are perhaps less commonly known in
object-oriented programming.  The interface of a monoid is defined as follows:
\begin{quote}
\lstinputlisting[linerange=5-8]{../ObjectAlgebras/src/expDemo3/Monoid.java} % APPLY:linerange=MONOID
\end{quote}
Intuitively, the \lstinline{join()} method is
used to combine the information from substructures, and the
\lstinline{empty()} is an indicator of ``no information''.
%% The monoid interface also includes one additional operation
%% called \lstinline{fold}, which takes a list of \lstinline{R}
%% and returns an aggregated \lstinline{R}. Finally, note that
%% even though \lstinline{Monoid} is an interface, the
%% \lstinline{fold} operation is defined. This is a new feature
%% of Java 8 called default methods, which allow default implementations
%% of methods in interfaces.

Using the monoid operations alone, it is possible to write a ``generic
query''. Fig.~\ref{generic_query} shows how this is achieved. In
nodes that contain child nodes, such as \lstinline{Add}, the
information is aggregated using \lstinline{join}. In nodes that
contain other information, such as \lstinline{Var} and
\lstinline{Lit}, the query returns \lstinline{empty}.
This allows concrete queries to be implemented by overriding methods from multiple, different algebras.

\begin{figure}[tb]
\nocaptionrule
\lstinputlisting[linerange=4-9]{../ObjectAlgebras/src/expDemo3/ExpAlgQuery.java} % APPLY:linerange=GENERIC_QUERY
\caption{Generic queries using a monoid.}
\label{generic_query}
\end{figure}

\begin{comment}
And now we find everything goes in an easier way: we don't care about
what kind of query it is any more during the traversal. Despite
whether it asks for all the names of free variables or a printer
showing the hierarchy of an expression, at first we can simply
override the method \lstinline{m()}, which provides an instance of
\lstinline{Monoid}, in the return statement. As the next step, we only
need to override a few other methods to meet the requirements. This is
the progress, once we have such a template dealing with the traversal,
all query algebras can be addressed in a more concise way, which is
called the \textit{generic query algebra}.
\end{comment}

\subsection{Free Variables with Generic Queries}\label{subsec:solvingfreevars}

%% \begin{figure}[t]
%% \lstinputlisting[linerange=9-14]{../ObjectAlgebras/src/expDemo3/SetMonoid.java} % APPLY:linerange=FREEVARS_MONOID
%% \vspace{-.1in}
%% \caption{A monoid instance defined for sets.}
%% \label{freevars_monoid}
%% \end{figure}

The \lstinline{ExpAlgQuery} interface provides an alternative way to
define the free variables operation.
Since the result of free variables is a set, the monoid returned by \lstinline{m()} is an implementation of the \lstinline{Monoid} interface, where \lstinline{empty()} corresponds to the empty set, and \lstinline{join()} is implemented as union.
Using this monoid
the free variables operation is defined as follows:

%As an alternative way to handle the freeVars query, the query algebra
%is going to be a sub-interface of \lstinline{ExpAlgQuery}, the generic
%algebra, with generic type to be \lstinline{List<String>}.

\lstinputlisting[linerange=8-11]{../ObjectAlgebras/src/expDemo3/FreeVars.java} % APPLY:linerange=FREEVARS_WITH_MONOID

There are two important differences to the implementation in
Fig.~\ref{freevars_without_monoid}.  Firstly, the monoid to be used
needs to be specified.  Secondly only the case for variables needs to
be defined: the other cases are inherited from
\lstinline{ExpAlgQuery}.

\begin{comment}
\haoyuan{Begin: client code.}
Now one can do some experiments with this approach. First comes an example of the expression built up like this:

\lstinputlisting[linerange=11-13]{../ObjectAlgebras/src/expDemo3/ExpTest.java} % APPLY:linerange=GEN_EXP

The method \lstinline{genExp()} just gives the expression \lstinline{x + (y + 2)}. And then an instance of \lstinline{FreeVars} is passed to \lstinline{genExp()} to get the names of free variables.
The result given by the following code is \lstinline{[x, y]}, which is a set of strings.

\lstinputlisting[linerange=19-20]{../ObjectAlgebras/src/expDemo3/ExpTest.java} % APPLY:linerange=CLIENTCODE_FREEVARS

\haoyuan{End: client code. x + (y + 2) and [x, y] in code style?}
\end{comment}

%% At this point it is interesting to analyze the benefits and drawbacks
%% of this approach. A clear benefit is that we have to define fewer cases
%% in specific queries. For large structures, this is can amount to
%% significantly less code. However we have to create the generic
%% classes for each object algebra interface as well as to use (and
%% perhaps define) some monoids. The monoid structures tend to be
%% quite general, so they can be reused accross multiple queries. For example
%% the set monoid is useful for any query that returns sets, not just
%% free variables. Therefore such monoids are typically included in
%% libraries, and do not need to be defined for each query.
%% For the code for the generic query itself (\lstinline{ExpAlgQuery})
%% it is nice that the code can be reused accross multiple
%% queries, but it still has to be defined. Furthermore the definition still
%% consists of boring and tedious code.

%Clearly if
%there are only a few types of nodes (as in our minimal example)
%and a few operations, the effort does not seem worthwhile. But

\begin{comment}
But the result for an expression can only be a null list based on the
monoid. Thus in the freeVars query, furthermore, we expect the
variables to store their names into a list, and by using the monoid,
freeVars can be implemented. See Fig.~\ref{freevars_with_monoid}.


When the interface \lstinline{FreeVarsExpAlg} is used, an object of
the \lstinline{FreeVarsMonoid} is then created. As we can see, it is
needless for a user to write an exclusive traversal fully for a data
structure. Nothing but a monoid is required together with a few
methods being overwritten. And furthermore, a monoid can usually be
shared among query algebras with the same data type.
\end{comment}

%\subsection{Generic Queries in \name}

The traversal code in \lstinline{ExpAlgQuery} is entirely
mechanical and can be automatically generated.  This is precisely what
\name does. Annotating algebra interfaces, such as \lstinline{ExpAlg}, with the annotation \lstinline{@Algebra},
triggers automatic generation of generic query interfaces, such as \lstinline{ExpAlgQuery}.
%\name also provides implementatations of a number of common monoids over sets, lists, and numbers.
Fig.~\ref{queryTemplate} shows the general template in \name for an algebra \lstinline[mathescape=true]{Alg<X$_1$, $...$, X$_n$>}, with constructors \lstinline[mathescape=true]{f$_1$, $...$, f$_m$}.
Note that interface \lstinline{Alg}$_Q$ extends \lstinline{Alg} so that all type parameters are unified as type \lstinline{R}.
All arguments to a constructor \lstinline[mathescape=true]{f$_j$} are combined with \lstinline{join} from the monoid \lstinline{m()}. \hl{Whereas those arguments with primitive types, like \lstinline{int}, \lstinline{boolean} or \lstinline{String}, are ignored by default.}

\begin{figure}[t]
\nocaptionrule
\begin{lstlisting}[mathescape=true]
interface Alg$_Q$<R> extends Alg<R,$...$,R> {
  Monoid<R> m();

  default R f$_j$(R p$_1$, $...$, R p$_k$) {
    return  m().join(p$_1$, m().join(p$_2$, $...$, m().join(p$_{k-1}$, p$_k$)$...$)));
  }
  $...$
}
\end{lstlisting}
\caption{Generic template for generating boilerplate of queries}
\label{queryTemplate}
\end{figure}

