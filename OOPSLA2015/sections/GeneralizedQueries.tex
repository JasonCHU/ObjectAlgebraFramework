\section{Generalized Queries}\label{sec:generalizedQueries}

The previous section introduced simple queries where each constructor
contributes to a single monoid.  Recursive data types, however, often
have multiple syntactic categories, for instance expressions and
statements.  In such multi-sorted Object Algebras each sort is
represented by a different type parameter in the algebra interface.
In this section we present \textit{generalized queries}, where each
such type parameter can be instantiated to different monoids.
It turns out that, for some operations, this generalized version
of queries is needed.

\begin{comment}
\haoyuan{General template. Some redundant texts.}
Figure~\ref{gQueryTemplate} shows the general template for \lstinline{Alg<X1, X2, ..., Xn>}.
Note that interface \lstinline{Alg}$_{GQ}$ extends \lstinline{Alg} and declares one monoid for each sort. As we cannot directly join a monoid over one type with a monoid over another type, for every constructor in \lstinline{Alg}$_{GQ}$, the arguments with different types from its return type are ignored.

\begin{figure}[t]
\begin{lstlisting}[mathescape=true]
interface Alg$_{GQ}$<X$_1$,$...$,X$_n$> extends Alg<X$_1$,$...$,X$_n$> {
  Monoid<X$_1$> m$_1$(); $...$; Monoid<X$_n$> m$_n$();

  default X$_i$ f$_j$(X$^1_p$ p$_1$, $...$, X$^k_p$ p$_k$) {
    X$_i$ res = m$_i$().empty();

    /* For all $s$ such that X$^s_p$ = X$_i$. */
    res = m$_i$().join(res, p$_s$);
    $...$
  }
  $...$
}
\end{lstlisting}
\caption{Generic template for generating boilerplate of generalized queries}
\label{gQueryTemplate}
\end{figure}
\end{comment}


\subsection{Example: data dependencies}
\label{subsec:depGraph}
%\subsection{Dealing with multiple sorts}

A simple example of a generalized query is the extraction of the data
dependencies between assignment statements and variables in simple
imperative programs.  To express this query, the simple
\lstinline{ExpAlg} is first extended with statements using the
\lstinline{StatAlg} interface defined as follows:

\lstinputlisting[linerange=7-10]{../ObjectAlgebras/src/trees/StatAlg.java} % APPLY:linerange=STATALG

The extension consists of statement constructors for sequential composition
(\lstinline{Seq}) and assignment (\lstinline{Assign}).  The generated
default implementation of queries over statements is shown in
Fig.~\ref{LST:gstatquery}\hl{, while the generated code for expressions (\lstinline{G_ExpAlgQuery}) is
presented in Appendix~\ref{subsec:appendix_code_g_expalgquery}}.  Note that the interface declares two
monoids, one for each sort.  Since the \lstinline{Assign} and
\lstinline{Seq} constructors create statements, they return elements
of the \lstinline{mStat()} monoid.  Furthermore, because it is
impossible to automatically join a monoid over one type with a monoid
over another type, the \lstinline{e} argument in \lstinline{Assign}
is ignored.  As a result, a
concrete implementation has to override this case to deal with the
transition from expressions to statements.

\begin{figure}[t]
\nocaptionrule
\lstinputlisting[linerange=7-16]{../ObjectAlgebras/src/deps2/G_StatAlgQuery.java} % APPLY:linerange=GSTAT_QUERY
\caption{Default implementation of queries over many-sorted statement algebra}
\label{LST:gstatquery}
\end{figure}


%An example of a query over the multi-sorted statement algebra is the extraction of data dependencies from a program.

Data dependencies are created by assignment statements: for a
statement \lstinline{Assign(String x, Exp e)} method, the variable
\lstinline{x} will depend on all variables appearing in \lstinline{e}.
The result of extracting such dependencies can be represented as
binary relation (a set of pairs). In expressions we need to collect
the free variables, which can be stored in a set of strings.  Thus in
this traversal two monoids are involved: a monoid for a set of pairs
of strings; and a monoid for a set of strings.

To implement the extraction of data dependencies only two cases have to
be implemented: the variable (\lstinline{Var}) case from the
\lstinline{ExpAlg} signature; and the assignment (\lstinline{Assign})
case from the \lstinline{StatAlg} signature.  The implementation is
shown in Fig.\ref{deps2}.  Note that the \lstinline{Assign} case takes
the input \lstinline{Set<String> e} and uses it to create the
dependency relation.  The propagation of dependencies across
sequential composition is automatic, as is the propagation of the set
of variables through the different types of expressions.


\begin{figure}[t]
\nocaptionrule
\lstinputlisting[linerange=12-23]{../ObjectAlgebras/src/deps2/DepGraph.java} % APPLY:linerange=DEPS2
\caption{Dependency Graph with Generalized Query Algebra}
\label{deps2}
\end{figure}

\begin{comment}
\haoyuan{Begin: client code.}

\lstinputlisting[linerange=14-17]{../ObjectAlgebras/src/deps2/TestDepGraph.java} % APPLY:linerange=GEN_DEPGRAPH

\lstinputlisting[linerange=23-27]{../ObjectAlgebras/src/deps2/TestDepGraph.java} % APPLY:linerange=CLIENTCODE_DEPGRAPH

\haoyuan{End: client code.}
\end{comment}

%The difference between generalized queries and homogeneous queries is that each type parameter of an Object Algebra interface is possibly instantiated to a different monoid.
%If each parameter is instantiated to the same monoid, generalized queries are equivalent to the queries of the previous section, with one important difference: in the latter case, each sort is contributing to the same monoid.
%The default implementation thus can carry over, for instance, query results of expression into the result of statements.
