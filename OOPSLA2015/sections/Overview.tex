\section{Overview}\label{subsec:overview}

This section starts by motivating the problem of boilerplate code in
traversals of complex structures. It then shows how \name addresses
the problem using a combination of object algebras~\cite{bruno12oa}
and Java annotations.

\subsection{A Questionnaire Structure and an Object Oriented Solution}

\haoyuan{Duplicate texts.}\\
\haoyuan{One reference is missing.}\\
\haoyuan{Style of presenting Java code.}

We start by considering the questionnaire structure. The example is actually closely based on our case study, but appears in a much more simplified form to illustrate our work clearly. To start with, a questionnaire is rendered as an interactive form where, depending on user actions, new questions may appear, or values may be computed. A simplified DSL for questionnaires (QL), has its syntax in our example with a data structure. Three sorts of components, Form, Stmt and Exp help to compose questionnaires.

In the full QL structure, an identified \lstinline{Form} comprises a list of \lstinline{Statements}. Each \lstinline{Statement} can either be a question (a variable depending on users' input or assigned with an \lstinline{Expression}) or a condition statement. Furthermore, an \lstinline{Expression} is based on several primitive types, and is obtained from certain values, variables, together with some possible arithmetic operations. In this section, the simplified version of QL has only six constructors.

\begin{figure}[t]
\nocaptionrule
\begin{lstlisting}[language=ql]
form DriverLicense {
  name: "What is your name?" string
  age: "What is your age?" integer
  if (age >= 18) {
    license: "Do you have a driver's license?" boolean
  }
}
\end{lstlisting}
\caption{Example QL questionnaire: driver's license}
\label{driver_license}
\end{figure}

A concrete example QL questionnaire is shown in Fig.~\ref{driver_license}, regarding driver's license. The questionnaire firstly asks for a user's name and age, and for an adult, furthermore, it asks whether the user has a driver's license.

Two operations are supported by the QL structure: collecting the set of used variables in the questionnaire; and renaming all variables with an underscore ``\_'' appended to the end. To support those operations, an OO solution is to have methods \lstinline{usedVars}, which returns a set of strings representing used variables, and \lstinline{rename}, which updates the names of all variables in a questionnaire.

\begin{figure}[t]
\nocaptionrule
\lstinputlisting[linerange=105-154]{../ObjectAlgebras/src/example_QLAlg1/QL.java} % APPLY:linerange=OO_APPROACH_SIMP
\caption{Java implementation of an OO approach for used variables and renaming}
\label{ql_oo_approach}
\end{figure}

Fig.~\ref{ql_oo_approach} shows the Java implementation of this OO approach. In the implementation, collecting the set of used variables is done by returning the variable name in instances of the \lstinline{Var} class, and delegating the method \lstinline{usedVars} to the child nodes in other classes. Renaming variables asks the instances of the \lstinline{Question} and \lstinline{Var} classes to do the renaming task, and similarly, other classes just delegates the method \lstinline{rename}. The renaming operation updates some information stored in the private members of those classes.

However, this simple OO solution has two important problems:

\begin{enumerate}

\item {\bf Lack of extensibility} The OO style
  representation of data structures is cumbersome and
  inflexible due to the bound relationship between classes. For
  instance, adding a new operation such as pretty
  printing the QL structure requires pervasive changes across all
  existing classes.

\item {\bf Boilerplate traversal code} Another issue is that we
  usually need to write a large amount of boilerplate code to
  implement traversals. In our example, we implemented
  \lstinline{usedVars} and \lstinline{rename} in all
  classes. However only the \lstinline{Question} and \lstinline{Var} classes do some
  interesting work. All other classes simply delegate the task to the
  child nodes. This problem becomes more severe with bigger tree
  structures. The interesting code can become a very small portion of
  the code needed to perform a traversal: most of the code is doing
  tedious walking of the structure.

\end{enumerate}

\subsection{Modeling the QL Structure with Object Algebras}

Object algebras~\cite{bruno12oa} provide a solution to the extensibility problem.
Object algebras are a design pattern that is useful to model complex
structures, while providing a lot of flexibility in terms of
modularity and extensibility. In particular, object algebras allow
extensibility in two dimensions: it is easy to add new types of
nodes; and new operations over the structure.

\begin{figure}[t]
\nocaptionrule
\lstinputlisting[linerange=7-15]{../ObjectAlgebras/src/trees/QLAlg.java} % APPLY:linerange=QL_TREE
\caption{QL structure represented by an object algebra interface}
\label{ql_tree}
\end{figure}

Fig.~\ref{ql_tree} shows the approach to model the QL
structure as an object algebra interface. Each kind of node (\lstinline{Exp}, \lstinline{Stmt}, \lstinline{Form})
is represented as a type parameter. Each method in the object algebra interface models a
constructor of some node in the QL structure. For example the
method \lstinline{form} constructs an instance of a form; whereas the
method \lstinline{question} creates an instance of a statement. From an
object-oriented perspective the methods in the interface are factory
methods. For the reader familiar with functional programming, the
resemblance to constructors of  (a system of) algebraic data types should be clear.

Operations are defined by implementing the object algebra interface.
The following code show an implementation of the used variables
operation.

\lstinputlisting[linerange=46-55]{../ObjectAlgebras/src/example_QLAlg2/UsedVars.java} % APPLY:linerange=QL_QUERY_ALG_SIMP

The \lstinline{UsedVars} class provides an implementation for each
of the methods in the object algebra interface, which defines the
overall used variables operation. Since the result of collecting those
variables is \lstinline{Set<String>}, all the type parameters are set to
that type. Most of the method implementations simply traverse the
child nodes and accumulate the salaries. That is the case, for
example, for \lstinline{form}. The only method implementation that does
something different is \lstinline{var}, which returns the salary
argument.

For the renaming operation the result is itself a structure
with all variable names updated. The following code shows the
implementation:

\lstinputlisting[linerange=36-45]{../ObjectAlgebras/src/example_QLAlg2/Rename.java} % APPLY:linerange=QL_TRANSFORM_ALG_SIMP

The class \lstinline{Rename} is parameterized by three types,
which represent the kinds of nodes in the QL structure. Each
constructor needs to build an instance of the right type of nodes.
Note that, due to the lack of better type-inference in Java, there is
some repetition of type-annotations. In order to create the updated
QL structure another algebra called \lstinline{alg} is used.
Almost all the method implementations reconstruct the structure with
no changes using the methods in \lstinline{alg}. Two exceptions are the
methods \lstinline{question} and \lstinline{var}, where the structure is also reconstructed, but
the names are modified with ``\_'' appended to the end.

Although we solved the problem of extensibility with object
algebras, the traversal code is still lengthy and we are still writing
tedious traversal code. In those two operations, the only interesting code
is in the methods \lstinline{question} and \lstinline{var}. Ideally, we would like to write only
the code for the interesting cases, and somehow ``inherit'' the
remainder tedious traversal code.

\subsection{\Name: An Object Algebra Framework for Traversals}

To deal with the boilerplate problem we created \Name: a Java object
algebras framework, which provides a number of generic traversals at
the cost of a simple annotation. The key idea in \name is to
automatically create highly generic object algebras, which encapsulate
common types of traversals. In particular \name supports generic
\emph{queries} and \emph{transformations}. Those two types of
traversals are useful to capture, respectively, the used variables and
renaming operations.

\begin{figure}[t]
\nocaptionrule
\lstinputlisting[linerange=10-17]{../ObjectAlgebras/src/example_QLAlg3/UsedVars.java} % APPLY:linerange=QL_QUERY_WITH_OAFRAMEWORK
\lstinputlisting[linerange=7-16]{../ObjectAlgebras/src/example_QLAlg3/Rename.java} % APPLY:linerange=QL_TRANSFORM_WITH_OAFRAMEWORK
\vspace{-.1in}
\caption{Used variables and renaming with \Name.}
\label{ql_with_oaframework}
\end{figure}

With \Name, programmers just need to add the \lstinline{@Algebra} annotation
to the definition of \lstinline{QLAlg} to get the code for generic
queries and transformations. An example of that annotation is already
shown in Fig.~\ref{ql_tree}. With that annotation several classes
are generated automatically, including \lstinline{QLAlgQuery} and
\lstinline{QLAlgTrans}. Using those classes the code needed to write
the used variables and renaming object algebras is much
shorter. The code for the two operations is shown in
Fig.~\ref{ql_with_oaframework}. By implementing the
\lstinline{QLAlgQuery} and \lstinline{QLAlgTrans} classes, only the methods
\lstinline{question} and \lstinline{var} needs to be overridden: all the other methods,
which do a simple generic traversal, are inherited. For queries the
only extra thing a programmer has to do is to provide an instance of a
monoid, which is used to specify how to accumulate the results during
the traversal. Similarly, for transformations, the programmer needs to
pass an algebra for providing the constructors for the transformation
code.

Some client code is shown as follows: %, using those two algebras based on our framework.

\lstinputlisting[linerange=11-22]{../ObjectAlgebras/src/example_QLAlg3/Test.java} % APPLY:linerange=CLIENTCODE_MAKEQL

This method is used to create a particular QL structure, in the
usual object algebras style.
Using this QL structure, we can use the \lstinline{UsedVars}
and \lstinline{Rename} to compute some information about the questionnaire:

\lstinputlisting[linerange=27-31]{../ObjectAlgebras/src/example_QLAlg3/Test.java} % APPLY:linerange=CLIENTCODE_QLTEST

The results give \lstinline{[age]} and \lstinline{[age_]}, which
is respectively the set of used variables before and after renaming.

The remainder of the paper provides the details and implementation
techniques used in \Name. Besides basic queries and transformations,
\name also supports two generalizations of these types of traversals
called \emph{generalized queries} and \emph{contextual transformations}.

