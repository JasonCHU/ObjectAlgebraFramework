\section{An Overview of \name}\label{subsec:overview}

This section starts by illustrating the problem of boilerplate code when implementing
traversals of complex structures. It then shows how \name addresses
the problem using a combination of Object Algebras~\cite{bruno12oa}
and Java annotations.

\subsection{Traversing Object-Oriented ASTs  }

%\haoyuan{Duplicate texts.}\\
%\haoyuan{One reference is missing.}\\
%\haoyuan{Style of presenting Java code.}


We start by introducing the problem of boilerplate code by considering
a simplified variant of the QL language used in our case
study~\cite{gouseti14extensible}, called MiniQL.  Just like QL, MiniQL
can be used to describe interactive questionnaires. An example is
shown in Fig.~\ref{driver_license}.  The questionnaire first asks
for the user's name and age, and then, if the age is greater than or
equal to 18, asks if the user has a driver's license.  Because of the
conditional construct, the last question only will appear when the
user is actually eligible to have driver's license.

MiniQL's abstract syntax contains forms, statements (if-then and question) and expressions (only literals, variables and greater-than-or-equal).
A traditional OO implementation is shown on the left Fig.~\ref{LST:usedVars}.
A form (class \lstinline{Form}) has a name and consists of a list of statements.
Statements are conditionals (\lstinline{If}) which contain an expression and a statement body, and questions (\lstinline{Question}) which have a name, label and type. Expressions are standard, but limited to literals (\lstinline{Lit}), variables (\lstinline{Var}) and greater-than-or-equal (\lstinline{GEq}).

\begin{figure}[t]
\nocaptionrule
\begin{lstlisting}[language=ql]
form DriverLicense {
  name: "What is your name?" string
  age: "What is your age?" integer
  if (age >= 18)
    license: "Have a driver's license?" boolean
}
\end{lstlisting}
\caption{Example QL questionnaire: driver's license}
\label{driver_license}
\end{figure}

%In the full QL structure, an identified \lstinline{Form} comprises a list of \lstinline{Statements}. Each \lstinline{Statement} can either be a question (a variable depending on users' input or assigned with an \lstinline{Expression}) or a condition statement. Furthermore, an \lstinline{Expression} is based on several primitive types, and is obtained from certain values, variables, together with some possible arithmetic operations. In this section, the simplified version of QL has only six constructors.

The code in Fig.~\ref{LST:usedVars} also shows a query over MiniQL structures, namely the collection of used variables.
The operation is defined using the method \lstinline{usedVars}, declared in the abstract superclasses  \lstinline{Stmt} and \lstinline{Exp} (omitted for brevity), and implemented in the concrete statement and expression classes.
As can be seen, the only interesting bit of code is the \lstinline{usedVars} method in class \lstinline{Var}.
All other implementations merely deal with aggregating results of their child nodes, or returning a default empty set.

The boilerplate code exhibited in the \lstinline{usedVars} query often
also applies to transformations.  Consider for example a rename
transformation which takes a \lstinline{Form} and returns another form
where all variables are renamed.  Again, the only interesting cases would be 
in the \lstinline{Var} and lstinline{Question} classes, where the actual renaming is
applied. All other classes, however, require boilerplate to recreate
the structure. The full code in
Appendix~\ref{subsec:appendix_code_oo_approach} contains a simple
example of such a rename operation as well.

%Two operations are supported by the QL structure: collecting the set of used variables in the questionnaire; and renaming all variables with an underscore ``\_'' appended to the end. To support those operations, an OO solution is to have methods \lstinline{usedVars}, which returns a set of strings representing used variables, and \lstinline{rename}, which updates the names of all variables in a questionnaire.
%\haoyuan{Clarify the differences between our solution and a usual OO approach.}

\begin{figure*}[t]
  \nocaptionrule
  \begin{minipage}[t]{0.48\textwidth}
\lstinputlisting[linerange=120-160]{../ObjectAlgebras/src/example_QLAlg1/QL.java} % APPLY:linerange=OO_APPROACH_SIMP
  \end{minipage}
  \vline
  \hspace*{2pt}
  \begin{minipage}[t]{0.48\textwidth}
  \lstinputlisting[linerange=50-82]{../ObjectAlgebras/src/example_QLAlg2/UsedVars.java} % APPLY:linerange=QL_QUERY_ALG_SIMP
  \end{minipage}

    \caption{Implementing the ``used variables'' operation using traditional ASTs (left) and Object Algebras (right)}\label{LST:usedVars}
\end{figure*}

%Fig.~\ref{ql_oo_approach} shows the Java implementation of this OO approach. In the implementation, collecting the set of used variables is done by returning the variable name in instances of the \lstinline{Var} class, and delegating the method \lstinline{usedVars} to the child nodes in other classes. Renaming variables asks the instances of the \lstinline{Question} and \lstinline{Var} classes to do the renaming task, and similarly, other classes just delegates the method \lstinline{rename}. The renaming operation updates some information stored in the private members of those classes.

In addition to the significant amount of boilerplate code, there is
another drawback to the traditional OO solution, which is that it does
not support extensibility along the dimension of operations. Each new
operation requires pervasive changes across the AST classes.

% the problem is exacerbated by the boilerplate...


\begin{figure}[t]
\nocaptionrule
\lstinputlisting[linerange=7-15]{../ObjectAlgebras/src/trees/QLAlg.java} % APPLY:linerange=QL_TREE
\caption{Object Algebra interface of the MiniQL abstract syntax}
\label{ql_tree}
\end{figure}

\subsection{Modeling MiniQL with Object Algebras}\label{subsec:model_ql_with_oa}


The right-hand side of Fig.~\ref{LST:usedVars} shows the used
variables operation implemented using Object Algebras. The operation
is a class implementing the Object Algebra interface
(\lstinline{QLAlg}) shown in Fig.~\ref{ql_tree}.

%Appendix~\ref{subsec:appendix_code_usedvars} and \ref{subsec:appendix_code_rename} gives the complete code for an implementation.}

The \lstinline{UsedVars} class provides an implementation for each
of the methods in the object algebra interface, which together define the
full used variables operation. Since the result of collecting those
variables is \lstinline{Set<String>}, all the type parameters are set to
that type. Most of the method implementations simply traverse the
child nodes and accumulate the variable names. That is the case, for
example, for \lstinline{Form}. Again, the only method implementation that does
something different is \lstinline{Var}, which returns the \lstinline{x}
argument.

%Although the implementation still exhibits the same amount of
%boilerplate code as the traditional AST-based implementation, 
Unlike the standard OOP implementation, 
Object Algebras support adding operations without changing existing code.
For instance, the renaming operation mentioned above could be realized
as follows\footnote{full code in
  Appendix~\ref{subsec:appendix_code_rename}}:

\lstinputlisting[linerange=36-47]{../ObjectAlgebras/src/example_QLAlg2/Rename.java} % APPLY:linerange=QL_TRANSFORM_ALG_SIMP

Each
constructor reconstructs a new node using an auxilar MiniQL algebra  called \lstinline{alg}.
Almost all the method implementations reconstruct the structure with
no changes using the methods in \lstinline{alg}. For instance, the \lstinline{Form} method just recreates the form in the algebra \lstinline{Alg}.  The two exceptions are the
methods \lstinline{Question} and \lstinline{Var}, where the identifiers are suffixed with ``\_''.

Although the Object Algebra encoding of MiniQL solves the problem of extensibility, the traversal code still contains boilerplate code.
In both \lstinline{UsedVars} and \lstinline{Rename}, the only interesting code
is in a small number of cases.  Ideally, we would like to write only
the code for the interesting cases, and somehow ``inherit'' the
tedious traversal code.
%\name  generates base algebra interfaces to support exactly that.

\subsection{\Name: An Object Algebra Framework for Traversals}

To deal with the boilerplate problem we created \Name: a Java Object
Algebras framework, which provides a number of generic traversals at
the cost of a single annotation. The key idea in \name is to
automatically create highly generic Object Algebras, which encapsulate
common types of traversals. In particular \name supports generic
\emph{queries} and \emph{transformations}. The two types of
traversals are, for instance,  sufficient to capture the used variables and
renaming operations.

\begin{figure}[t]
\nocaptionrule
\lstinputlisting[linerange=10-17]{../ObjectAlgebras/src/example_QLAlg3/UsedVars.java} % APPLY:linerange=QL_QUERY_WITH_OAFRAMEWORK
\caption{MiniQL used variables, implemented with \Name.}
\label{usedvars_with_oaframework}
\end{figure}

\begin{figure}
\nocaptionrule
\lstinputlisting[linerange=7-15]{../ObjectAlgebras/src/example_QLAlg3/Rename.java} % APPLY:linerange=QL_TRANSFORM_WITH_OAFRAMEWORK
%\vspace{-.1in}
\caption{MiniQL renaming, implemented with  \Name.}
\label{rename_with_oaframework}
\end{figure}

With \Name, programmers just need to add the \lstinline{@Algebra} annotation
to the definition of \lstinline{QLAlg} to get the code for generic
queries and transformations. An example of that annotation is already
shown in Fig.~\ref{ql_tree}.
Triggered by the annotation, \name generates base traversal interfaces with Java 8 \lstinline{default} methods which can then be overridden to implement specific behavior.
For instance, for the MiniQL algebra, \name generates interfaces
\lstinline{QLAlgQuery} and
\lstinline{QLAlgTrans} which can be used to implement \lstinline{UsedVars} and \lstinline{Rename} in only a fraction of the code\footnote{The generated code is available in Appendix~\ref{subsec:appendix_code_qlalgquery} and \ref{subsec:appendix_code_qlalgtransform}.}.

The \name-based implementation of both operations is shown in
Fig.~\ref{usedvars_with_oaframework} and
\ref{rename_with_oaframework}. In contrast to Fig.~\ref{LST:usedVars}, the 
code in Fig.~\ref{usedvars_with_oaframework} is much shorter. 
By implementing the
\lstinline{QLAlgQuery} and \lstinline{QLAlgTrans} interface, only the
methods \lstinline{Question} and \lstinline{Var} need to be
overridden: all the other methods perform basic accumulation for
queries and basic identity reconstruction in the case of
transformations.  For queries the only extra thing a programmer
has to do is to provide an instance of a monoid, which is used to
specify how to accumulate the results during the traversal. Similarly,
for transformations, the programmer needs to pass an algebra for
providing the constructors for creating the result of a
transformation.

\paragraph{Client code} 
To use the queries and operations on a questionnaire like the one in Fig.~\ref{driver_license}, we need a function to create a structure using the generic MiniQL interface:

\lstinputlisting[linerange=9-17]{../ObjectAlgebras/src/example_QLAlg3/Test.java} % APPLY:linerange=CLIENTCODE_MAKEQL

Since both queries and transformations are implementations of the MiniQL interface, they can be passed to the \lstinline{makeQL} function defined above:
\lstinputlisting[linerange=26-27]{../ObjectAlgebras/src/example_QLAlg3/Test.java} % APPLY:linerange=CLIENTCODE_QLTEST

This code prints out  \lstinline{[age]} and \lstinline{[age_]}, which
are the set of used variables before and after renaming, respectively.
Note how the \lstinline{Rename} transformation transforms the questionnaire into the \lstinline{UsedVars} algebra.

The remainder of the paper provides the details and implementation
techniques used in \Name. Besides basic queries and transformations,
\name also supports two generalizations of these types of traversals
called \emph{generalized queries} and \emph{contextual transformations}.

