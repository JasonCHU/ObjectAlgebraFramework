\section{Transformations}\label{sec:transformations}
\label{retroimp}

Queries are a way to extract information from a data structure.
Transformations, on the other hand, allow data structures to be changed.
Just as with queries, we can distinguish code that deal with traversing the data structure from code that actually changes the structure.
In this section we show how to avoid most traversal boilerplate code in the context of transformations based on Object Algebras.


\subsection{Variable substitution: a simple tranformation algebra}\label{subsec:substvars}

A simple example of a transformation algebra, based on the interface \lstinline{ExpAlg} is substituting expressions for variables. A manual implementation based on Object Algebras is shown in Fig.~\ref{substvars_without_id}.

\begin{figure}[!htbp]
\lstinputlisting[linerange=4-10]{../ObjectAlgebras/src/expDemo2/SubstVar.java} % APPLY:linerange=SUBSTVARS_WITHOUT_ID
\vspace{-.1in}
\caption{A normal algebra-based implementation of variable substitution}
\label{substvars_without_id}
\end{figure}

The expression to be substituted, and the variable to substitute for are computed by the methods \lstinline{e()} and  \lstinline{x()} respectively.
\lstinline{expAlg()} is an instance of \lstinline{ExpAlg} on which the transformation is based.
Since Object Algebras are factories, the transformation is executed immediately during construction.
For instance, calling \lstinline{Var("x")} on a \lstinline{SubstVar} object with \lstinline{x()} returning \lstinline{"x"} immediately returns the result of \lstinline{exp()}, --- the original variable expression is never created.
In the other cases, the original structure is recreated in the algebra \lstinline{expAlg()}.

Again we observe the problem of traversal-only boilerplate code: the \lstinline{Lit} and \lstinline{Add} methods simply delegate to the base algebra \lstinline{expAlg()} without performing any computation.
This boilerplate code can be avoided by creating a super-interface containing default methods performing the traversal (shown in Fig.~\ref{generic_transform}).
A concrete transformation can then selectively override the cases of interest.
Variable substitution can now be implemented as shown in Fig.~\ref{substvars_with_id}.
In this case, only the  method \lstinline{Var()} is overridden.


\begin{figure}[t]
\lstinputlisting[linerange=4-9]{../ObjectAlgebras/src/expDemo3/ExpAlgTransform.java} % APPLY:linerange=GENERIC_TRANSFORM
\vspace{-.1in}
\caption{Traversal-only base interface for implementing transformations of expressions}
\label{generic_transform}
\end{figure}


\begin{figure}[t]
\lstinputlisting[linerange=4-7]{../ObjectAlgebras/src/expDemo3/SubstVar.java} % APPLY:linerange=SUBSTVARS_WITH_ID
\vspace{-.1in}
\caption{The transformation interface for substVars}
\label{substvars_with_id}
\end{figure}


To execute substitution, the \lstinline{SubstVar} should be subclassed (either anonymously or explicitly) with implementations for \lstinline{x()}, \lstinline{e()}, and \lstinline{expAlg()}.
The base algebra \lstinline{expAlg()} could for instance be an algebra for pretty printing the expression with the substitution applied.
Note that this allows pipelining of transformations: there is no reason  \lstinline{expAlg()} can not return yet another transformation algebra.

\haoyuan{Begin: client code.}

Here we also give the client code. With the same expression \lstinline{x + (y + 2)}, a \lstinline{SubstVar} object is created. The base algebra \lstinline{expAlg()} here is implemented with an instance of \lstinline{FreeVars}, and we also override \lstinline{x()} and \lstinline{e()} aiming to substitute another variable \lstinline{"z"} for \lstinline{"x"}. The code is as follows, where the result of execution becomes \lstinline{[y, z]}.

\lstinputlisting[linerange=28-33]{../ObjectAlgebras/src/expDemo3/ExpTest.java} % APPLY:linerange=CLIENTCODE_SUBSTVAR

\haoyuan{End: client code.}


%Another important characteristic is that we can apply several transformations to the data structure before a query. This pattern is called the \textit{transformation pipeline}. Since the transformation algebra \lstinline{ExpAlgTransform<Exp>} is a sub-interface of \lstinline{ExpAlg<Exp>}, it creates any instance with the same type as the \lstinline{expAlg()} inside. Therefore a programmer can define a number of transformation algebras, and compose them in sequence, like a nested structure. At the meantime, a query algebra is passed to the innermost \lstinline{expAlg()}. In that case, the query is traversed recursively throughout the pipeline and eventually derives a certain composite transformation algebra.


%The \lstinline{SubstVarsExpAlg} is still generic, yet something like a pretty printer can be passed to the \lstinline{expAlg()} inside as a query, to display the results. Hence at this moment, a programmer doesn't need to write boilerplate code for traversals. The identity approach and the pipeline of transformations provide users with a generic transformation like a template.

\subsection{Extensible Transformations}

\name transformations are extensible in the same way that \name queries are extensible.
To illustrate extension of transformation, consider the simple transformation that makes all variable occurrences unique:

%% weird, computeposition does not compute the right range.
\lstinputlisting[linerange=4-7]{../ObjectAlgebras/src/expDemo3/Unique.java} % APPLY:linerange=UNIQUEVARS

This transformation uses a helper method \lstinline{nextInt()} which returns consecutive integers on each call.
The basic transformation simply renames \lstinline{Var} expressions.
If the expression language is extended with lambda constructs, the transformation needs to be extended as well.

\lstinputlisting[linerange=4-6]{../ObjectAlgebras/src/expDemo3/UniqueWithLambda.java} % APPLY:linerange=EXTEND_UNIQUEVARS

Note that the transformation uses the \lstinline{lamAlg()} algebra (generated in \lstinline{LamAlgTransfrom}),
to create lambda expressions.


%% Alternatively, a single method would suffice in the single-sorted case, since sub-classes or interfaces may refine the return type. 

%% \begin{lstlisting}[mathescape=true]
%% interface UniqueWithLambda<E, Alg extends ExpAlg<E> & LamAlg<E>>
%%     extends Unique<E>, LamAlgTransform<E> {	
%%   Alg expAlg();
  
%%   default E Lam(String x, E e) { return expAlg().Lam($...$); }
%% }
%% \end{lstlisting}
