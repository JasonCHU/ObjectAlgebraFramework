\section{Transformations}\label{retroimp}

Here's an example of using code that gets imported from the code repository:

\begin{figure}[tb]
\lstinputlisting[linerange=23-38]{../src/finalcode/Retroactive.java} % APPLY:linerange=RETROACTIVE
\vspace{-.1in}
\caption{A retroactive implementation of printing for arithmetic expressions.\label{retroactive}}
\end{figure}

The way the code gets imported from the repository is as follows:

\begin{itemize}

\item We annotate the code in the repository with comments like: ``//BEGIN\_RETROACTIVE''
  and ``//END\_RETROACTIVE''.

\item We change the latex code (i.e. this file) so that
  ``APPLY:linerange='' points to the right label. In this case, the
  label is ``RETROACTIVE''.

\item We run the ruby script ``computePositions.rb'' using ``ruby
  computePositions.rb'' in the command line.

\item Finally we run ``pdflatex paper.tex'' and see if everything
  worked as expected.

\end{itemize}

Some more examples of code:

\begin{lstlisting}[numbers=none]
Exp exp = new Add(new Lit(3), new Lit(4));
\end{lstlisting}

\noindent we would abstract uses of the constructors as follows:

\begin{lstlisting}[numbers=none] 
<A> A exp(IntAlg<A> v) {
  return v.add(v.lit(3), v.lit(4));
}
\end{lstlisting}