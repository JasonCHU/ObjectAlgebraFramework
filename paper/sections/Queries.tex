\section{Queries}\label{sec:queries}

As a specific type of object algebras, queries allow users to define
new operations handling a user-defined data structure\bruno{This
  definition of queries is just too broad. A query is an operation
  that traverses a structure and computes some aggregate value. Please
look at papers like Syb to see how they describe queries and transformations}. A \textit{query
  algebra} is a class implementing an object algebra interface by a
top-down traversal throughout the hierarchy. It is something
supporting the program to gather information from the substructures of
a data type recursively, and make a response at the root node to the
query.
\begin{comment}
\bruno{Haoyuan: please stop using ``$\backslash\backslash$" after every paragraph!
A normal paragraph is created by having an empty line, not by using
``$\backslash\backslash$". The ``$\backslash\backslash$" command is only for forcing a new line.}
\end{comment}
\begin{comment}
Write a new definition. Class or interface?
\end{comment}

\subsection{FreeVars: a simple query algebra}\label{subsec:freevars}

An example is shown here to discuss about query algebras in a clearer
way. The object algebra interface is related to an expression, where
it can be treated as a numeric literal, a variable or the addition of
two expressions. Specifically, the structure is defined as follows:

\begin{figure}[!htbp]
\lstinputlisting[linerange=6-11]{../ObjectAlgebras/src/trees/ExpAlg.java} % APPLY:linerange=EXP_ALG
\vspace{-.1in}
\caption{Object algebra interface: \lstinline{ExpAlg}}
\label{exp_alg}
\end{figure}
\begin{comment}
\bruno{Don't copy the code! Use the code from the
  repository directly via the ruby script! The same applies to the
  other pieces of code in the rest of the paper.}
\end{comment}

Based on the interface above, a query might be raised on collecting
all the names of free variables defined in an expression. More
precisely, a list of strings would be used to store the names of
those variables. In that case, a \lstinline{Var(s)} would simply return a singleton list of \lstinline{s}, and a \lstinline{Lit(i)} corresponds to an empty list, whereas two lists would be joined into one if we are combining two
expressions with the \lstinline{Add()} method.

Generally speaking, it is natural to deal with the traversal in an algebra-based approach like Fig.~\ref{freevars_without_monoid}.

\begin{figure}[tb]
\lstinputlisting[linerange=12-18]{../ObjectAlgebras/src/expDemo2/FreeVarsExpAlg.java} % APPLY:linerange=FREEVARS_WITHOUT_MONOID
\vspace{-.1in}
\caption{A normal algebra-based approach for freeVars}
\label{freevars_without_monoid}
\end{figure}
\begin{comment}
\bruno{put code in a figure, add a caption and refer
  to the figure in the text.}
\end{comment}

Information on our query is collected by traversal and passed on to a
higher-level structure. Nonetheless, a programmer has to write a lot
of boring code handling the traversals, and it could be even worse for
a more complicated data structure. Moreover, it is a query-based
approach: you still have to write a bunch of similar stuff with a
different query raised, for instance, a pretty printer.

\subsection{Generic query algebra with a monoid}\label{subsec:genericquery}

Queries are so similar actually: a user has to indicate the rules in
which the program may address cases on primitive types and ``append''
the information. With these two issues, everything becomes simple in
the traversal. Hence we introduce the concept of monoid and generic
traversal here in our query algebras.

\begin{figure}[tb]
\lstinputlisting[linerange=6-12]{../ObjectAlgebras/src/expDemo3/Monoid.java} % APPLY:linerange=MONOID
\vspace{-.1in}
\caption{A generalized monoid interface}
\label{monoid}
\end{figure}

The interface of a monoid is defined in Fig.~\ref{monoid}. Intuitively, the \lstinline{join()}
method implies how we combine the information from substructures during
merging, and the \lstinline{empty()} is just an indicator of ``no
information''. Hence now we are able to write a ``generic traversal''
manually based on monoids. See Fig.~\ref{generic_query}.
\begin{comment}
\bruno{As you can see, nothing
  follows! Please refer to the figure instead and briefly explain the
  code in the figure.}
\end{comment}

\begin{figure}[tb]
\lstinputlisting[linerange=4-20]{../ObjectAlgebras/src/expDemo3/ExpAlgQuery.java} % APPLY:linerange=GENERIC_QUERY
\vspace{-.1in}
\caption{Generic query by hand with monoid}
\label{generic_query}
\end{figure}

And now we find everything goes in an easier way: we don't care about what kind of query it is any more during the traversal. Despite whether it asks for all the names of free variables or a printer showing the hierarchy of an expression, at first we can simply override the method \lstinline{m()}, which provides an instance of \lstinline{Monoid}, in the return statement. As the next step, we only need to override a few other methods to meet the requirements. This is the progress, once we have such a template dealing with the traversal, all query algebras can be addressed in a more concise way, which is called the \textit{generic query algebra}.

\subsection{Solving freeVars with generic query algebra}\label{subsec:solvingfreevars}

As an alternative way to handle the freeVars query, the query algebra is going to be a sub-interface of \lstinline{ExpAlgQuery}, the generic algebra, with generic type to be \lstinline{List<String>}. To use the generic traversal code, a monoid is defined in Fig.~\ref{freevars_monoid}.

\begin{figure}[tb]
\lstinputlisting[linerange=8-17]{../ObjectAlgebras/src/expDemo3/FreeVarsMonoid.java} % APPLY:linerange=FREEVARS_MONOID
\vspace{-.1in}
\caption{A monoid instance defined for freeVars}
\label{freevars_monoid}
\end{figure}
\bruno{Is an array the best structure to use here?
  Wouldn't a vector or list be better? The simpler the code is, the better.}

But the result for an expression can only be a null list based on the monoid. Thus in the freeVars query, furthermore, we expect the variables to store their names into a list, and by using the monoid, freeVars can be implemented. See Fig.~\ref{freevars_with_monoid}.

\begin{figure}[tb]
  \lstinputlisting[linerange=7-14]{../ObjectAlgebras/src/expDemo3/FreeVarsExpAlg.java} % APPLY:linerange=FREEVARS_WITH_MONOID
\vspace{-.1in}
\caption{The query interface for freeVars}
\label{freevars_with_monoid}
\end{figure}

When the interface \lstinline{FreeVarsExpAlg} is used, an object of the \lstinline{FreeVarsMonoid} is then created. As we can see, it is needless for a user to write an exclusive traversal fully for a data structure. Nothing but a monoid is required together with a few methods being overwritten. And furthermore, a monoid can usually be shared among query algebras with the same data type.
