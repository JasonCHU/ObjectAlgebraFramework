\section{Related Work}\label{sec:related}

To the best of our knowledge, few work has been done addressing
writing generic queries and transformations with object algebras,
while we should mention some work on object algebras and large tree
structures traversal control, which inspired and formed the basis of
our work.

\paragraph{Adaptive Object-Oriented Programming (AOOP)}
In AOOP~\cite{DemeterBook} there is a domain-specific language for
selecting parts of a structure that should be visited. This is useful
to do traversals on complex structures and focus only on the
interesting parts of the structure relevant for computing the final
output.  DJ is an implementation of AOOP for Java using
reflection~\cite{DJ}. More recently, inspired by AOOP,
DemeterF~\cite{OOGP} improved on previous approaches by providing
support for safe traversals, generics and data-generic function
generation.  Compared to \name most AOOP approaches are not
type-safe. Only in DemeterF a custom type system was designed to
ensure type-safety of generic functions. However, in contrast to
\Name, DemeterF requires a new language and it is unclear how DemeterF
traversals interact with extensibility.

\paragraph{Structure-Shy Traversals with Visitors}. 
Visser\cite{visser01visitor} provided some visitor
combinators that can express interesting traversal strategies in
visitor pattern. We applies some similar idea like identity
transformation in simple transformation, but our work targets at
traversal control in Object Algebras.\bruno{Tijs}\bruno{More work: 
The essence of the visitor pattern, and follow-up work to improve 
performance.}

\paragraph{Strategic Programming}\bruno{Tijs}

\paragraph{Object Algebras.} Oliveira proposed Object Algebras as a
solution to Expression Problem. Object algebras applies well in
mainstream object oriented languages like Java. It is a lightweight
solution in terms of language features required. Oliveira also worked
on using intersection type and type-constructor polymorphism to make
object algebras compositional with feature oriented
programming. Different from their prior research, we found that when
applying object algebras in rich tree structures, boilerplate code is
hard to avoid. Hence our work focuses on reducing the amount of
boilerplate code for developers when writing queries and
transformations with object algebra tree structures.

\textit{Structure-Shy Traversals in Functional Programming.} 
In the functional programming community like Haskell, much research
has been done on traversal control of large structures. Lammel and
Peyton Jones' ``Scrap your Boilerplate''~\cite{ralf03syb,?,?} series
introduced a practical design pattern for doing type-safe
structure-shy traversals in tree structures, and were a source 
of inspiration. Quite different implementation techniques ... 
SyB traversals are notoriously slow due to the use of some run-time 
reflection techniques.
%The notions of queries 
%and transformations are inspired by their work.
Bringert\cite{bjorn08acf} introduced useful compositional
functions to help construct final results in
Haskell. Lammel\cite{ralf00banana} proposed a polytypic programming
approach for generalized and basic folds. These fold algebras scale up
applications involving large systems of mutually recursive
datatypes. These works all try to optimize traversal control of large
structures in functional programming paradigm, while our work solves a
similar problem in Object Algebras, a programming style in Object
Oriented Programming paradigm.

In summary, prior to our work, research has been done on object
algebras and composition problem of this programming style. In the
functional programming world and with visitor pattern, traversal
control in large structures is also explored. Different from these
work, we explored techniques helping write generic queries and
transformations traversing large tree structures with Object Algebras.

