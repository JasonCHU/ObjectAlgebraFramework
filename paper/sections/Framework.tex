\section{\name Implementation}

\name is implemented using the standard Java annotations framework
(\lstinline{javax.annotation}) packaged in the Java Development Kit.
All of the generic traversals discussed in the previous sections --
generic queries, generalized generic queries, generic transformations,
and contextual generic transformations -- are automatically generated
by \name for an object algebra interface annotated with "$@$Algebra".
Furthermore, \name includes the monoid interface as well as several
useful instances of the monoid interface.

The big advantage of using standard Java annotations is that the code 
generation of the generic traversals can be done transparently: users 
do not need to use or install a tool to generate that code. As a
result \name is as simple to use as a conventional library. 
Another advantage of \name is that it is quite easy to extend with new 
types of traversals. \bruno{Say something about this}

\bruno{Should we talk about code size? How big is the \name framework?} 

\begin{comment}
To address this problem, we provide a framework \name, which utilizes \emph{Java Annotation} to generate query and transformation interfaces based on the \emph{Object Algebra Interface}. For the below code: 
\begin{lstlisting}[numbers=none] 
@Algebra
public interface ExpAlg<Exp> {
	Exp Var(String s);
	Exp Lit(int i);
	Exp Add(Exp e1, Exp e2);
}
\end{lstlisting}
with the annotation "$@$Algebra", the framework will generate the boilerplate codes for us automatically. As for our ExpAlg example, the following directory structure will be generated by the library.

\dirtree{%
 .1 src/.
 .2 query/.
 .3 ExpAlgQuery.
 .3 G\_ExpAlgQuery.
 .2 transform/.
 .3 ExpAlgTransform.
 .3 G\_ExpAlgTransform.
}

Here the automatically generated ExpAlgQuery, G\_ExpAlgQuery, ExpAlgTransform and G\_ExpAlgTransform are exactly the same code as we discussed in the previous sections. Developers can implement interesting methods by inheriting from these generic classes without worrying about the traversing work.


With our framework \name, when programming with query and
transformations, the programmer can skip the intermediate steps such
as constructing generic queries and transformations, but only focus on
rewriting the interesting cases. For instance, in our ExpAlg example,
to implement FreeVars algebra, we can simply override the
\lstinline{Exp Var(String s)} method of \lstinline{class ExpAlgQuery}
to return variable name, and provide the specific monoid needed, which
in this case will be a \lstinline{StringListMonoid}. While the
SubstVars algebra can be realized by overriding the \lstinline{Exp
  Var(String s)} method of ExpAlgTransform interface, which
substitutes variable names as specified. 
\end{comment}
