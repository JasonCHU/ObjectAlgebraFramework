\section{Introduction}

Many applications require complex recursive data structures. Examples
abound, for example, in language processing tools/libraries for programming
languages, domain-specific languages, markup languages like HTML, or
data-interchange languages like XML or JSON. In those applications
Abstract Syntax Trees (ASTs) are the key data structure needed to
model the various constructs of the languages. Such ASTs have various
different types of nodes, which can range from a few dozen to
several hundred nodes (for example in the ASTs of languages like Java).

Static types are helpful to deal with such complex structures.  With
static types it is easy to distinghish between different kinds of
nodes. Furthermore the distinctions are helpful to ensure that
traversals over these structures have an appropriate piece of code that
deals with each different type of node. This can prevent a large class
of run-time errors that would not otherwise be detected.

%For example static types prevent errors that would arize from
%applying the code intended for a certain kind of node to another kind
%of node. Static types can also ensure that all nodes are dealt with by
%an appropriate piece of code.

Unfortunately, when traversing such structures, the number of nodes and the
enforced type distinctions between nodes can lead to so-called
\emph{boilerplate code}~\cite{ralf03syb}: code that is similar for most types of nodes and which
essentially walks the structure. Traversals where such boilerplate
code dominates are called \emph{structure shy}~\cite{DemeterBook}. In
structure shy operations only for a small portion of nodes the code
for the core functionality that motivated the traversal in the first
place will be different. A typical example is computing the free
variables of an expression for some programming language. In this
case, the interesting code occurs in the nodes representing the
binding constructs. In all other nodes, the code would just deal with
walking the structure. In data structures with dozens or hundreds of
nodes, having to explictly write cases for each node is a source of
significant complexity, since it requires a lot of effort and it is
error-prone.
%In summary, structure shy operations
%suffer from the static typing discipline, because the type
%distinctions between different kinds of nodes leads to significant
%boilerplate code.

The boilerplate problem when traversing complex structures has
received considerable attention in the past. For example, both
\emph{Adaptive Object-Oriented Programming} (AOOP)~\cite{DemeterBook}
and \emph{Strategic
  Programming}~\cite{borovansky1996elan,visser1998core,vandenBrand:2003:TRT:941566.941568}
are aimed partly at solving this problem. Most approaches to AOOP and
strategic programming use some meta-programming techniques, such as
code generation or reflection. The use of meta-programming offers
programmers an easy way to deal with the code for the nodes that are
not structure shy. This has important benefits. Firstly the user has
to write much less code, also removing the possibility of errors in
the code walking the structure. Secondly the code becomes much more
adaptive to changes: if structural changes occur only in the structure
shy cases, then the user-defined code remains unchanged.  However such
meta-programming based approaches are usually come at the cost of
other desirable properties, such as modular type-safety or
separate-compilation. The functional programming community has also
studied the problem before. Various approaches, such as the popular
``Scrap your boilerplate''~\cite{ralf03syb}, support type-safety and separate
compilation.  However most of the techniques used in functional
languages cannot be easily ported to OO languages like Java, and are
limited in terms of extensibility.

This paper presents a Java framework called \name that allows users to
define \emph{type-safe and extensible structure-shy operations}. \name uses
\emph{object algebras}~\cite{bruno12oa} to describe complex data structures. Object
algebras are a recently introduced technique, which has been shown to
have significant advantages for software extensibility.  In \name
object algebras interfaces are combined with Java annotations to
generate generic and reusable object algebras that deal with
boilerplate traversal code.  Those object algebras include different
types of traversals: \emph{queries}; \emph{transformations};
\emph{generalized queries}; and \emph{contextual transformations}. Programmers
that want to implement structure shy traversals can inherit the
generic traversal code, and focus only on writing the interesting
parts of the traversals. Consequently, the amount of code that
programmers need to write is significantly smaller, and traversals
written in \name are:

\begin{itemize}

\item {\bf Adaptive and structure shy:} \name traversals can omit
  boilerplate code, allowing these traversals to be more adaptive to
  future changes or extensions to the data structure.

\item {\bf Simple and general:} \name traversals work for any
  structure that can be expressed as a (multi-sorted) object
  algebra. This includes complex OO hierarquies or ASTs for large
  languages. Very often traversals are quite simple, being
  implementable in just a few lines of code, even for complex
  structures with hundreds of different types of nodes.

\item {\bf Implemented in plain Java:} \name traversals do not require
  a new tool or language. The approach is library based and uses only
  Java annotations. 

\item {\bf Type-safe and separately compilable:} \name traversals are directly written in Java
  and the Java type-system ensures type-safety. No run-time casts are
  needed for generic traversal code or for user-defined traversal
  code. Furthermore \name traversals support separate compilation. 

\item {\bf Extensible:} Traversals inherit type-safe
  extensibility from object algebras. Both traversals and structures
  are extensible. Therefore it is possible to
  reuse traversal code in structures that are extended with additional
  nodes.

%\item {\bf Performant:} Traversals written with \name have performance that 
%is comparable to naive hand-written OO code.

\end{itemize}

To prove the effectiveness of the approach, we employeed \name 
on the implementation of the domain-specific questionaire 
language QL~\cite{gouseti14extensible}. 
Our results show that for a large number of traversals 
there was a significant reduction in the amount of user-defined code.

Although we have chosen Java as the implementation language for \Name,
our approach should apply to any OO language with support for generics
and annotations. For example it should be easy to port \name to
languages such as Scala, or .Net languages.

\begin{comment}

Existing work: 

dynamic approaches: they can offer adaptivity and 
structure shyness, but lack type-safety.

static approaches: No fully type-safe approach that we know 
of for Java. Some approaches that rely on reflection and/or casts.
But these can have run-time type-errors and performance can be 
quite bad. 
Some static type-safe in the functional programming community, 
but cannot be easily ported to Java (rely on sophisticated type 
system features; target algebraic datatypes; ...). Moreover 
(with a few exceptions) most approaches do not offer extensibility. 

Static types are useful to distinguish between different kinds
of nodes in a structure and to prevent misuses. The
distinction between different types of nodes also means that code for
dealing with each type of node is needed. However, in large structures, 
such as the Abstract Syntax Tree (AST) of a programming language, 
the amount of required code can be a problem. For some operations, 
which traverse large structures, most of the code amounts to recursively 
delegating the traversal to the child nodes. Only for some nodes 
of the structure the code needs to do something different. Still 
the programmer needs to diligently and tediously write the error-prone 
traversal code for all nodes.

In this paper we present a framework 

This approach works well for many operations,
which need different (and non-trivial) code for each different type 
of node 

%Unfortunately, for some operations the interesting 

%%In large structures, 
%%such as the Abstract Syntax Tree (AST) of a programming language, 
%%the code required to traverse the whole structure is proportionaly large.


The problem is particularly prominent 
in statically typed languages, where the typing discipline 
enforces strict distinctions between different cases 

Operations that traverse complex structures often require large and 
tedious amounts of boilerplate code. In those operations there are 
typically a few 

A pervasive problem in programming occurs when large tree tr

\end{comment}

In summary, the contributions of this paper are:

\begin{itemize}

\item {\bf Design patterns for generic traversals.} We provide a set of design 
patterns for various types of traversals using object algebras. These include: 
\emph{queries}, \emph{transformations},
\emph{generalized queries} and \emph{contextual transformations}.

\item {\bf The Shy Java framework.} We have implemented\footnote{{\bf
      Note to Reviewers:} Due to the anonymous review process we
    submit a file bundle with the implementation together with the paper submission. If the
    paper is accepted we will make the code publicaly available.} a
  Java framework that can be used to describe complex structures using
  object algebras; and to eliminate boilerplate code. The framework
  uses Java annotations to automatically generate generic traversals.
 

\item {\bf Case study and empirical evaluation.} We evaluate the
  approach using a case study based on the QL domain-specific
  language. The results of our case study show significant savings in
  terms of user-defined traversal code.

\end{itemize}
