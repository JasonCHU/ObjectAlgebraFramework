\section{Introduction}

Static types can be quite helpful when dealing with complex structures.

Existing work: 

dynamic approaches: they can offer adaptivity and 
structure shyness, but lack type-safety.

static approaches: No fully type-safe approach that we know 
of for Java. Some approaches that rely on reflection and/or casts.
But these can have run-time type-errors and performance can be 
quite bad. 
Some static type-safe in the functional programming community, 
but cannot be easily ported to Java (rely on sophisticated type 
system features; target algebraic datatypes; ...). Moreover 
(with a few exceptions) most approaches do not offer extensibility. 


Our technique has the following properties:

\begin{itemize}

\item {\bf Adaptivity and structure shyness:}

\item {\bf Simple and general:} Works for any structure that can be expressed 
as a (multi-sorted) object algebras. This includes complex OO hierarquies
or AST for large languages. 

\item {\bf Works in plain Java:} There is no need for a new tool or language. 
The approach is library based and uses only Java annotations. 

\item {\bf Type-safe reuse:}   All reuse is type-safe. No run-time
  casts are needed for generic traversal code or for user-defined 
  traversal code. 

\item {\bf Extensibility:} The approach inherits type-safe
  extensibility from object algebras. Therefore it is possible to
  reuse traversal code in structures that are extended with additional
  constructors.

\end{itemize}

Moreover our approach should apply to any OO language with support 
for generics and annotations (for example: Scala, .Net?).

Traversing complex data structures typically requires large amounts of
tedious boilerplate code. For some operations most of the code simply
walks the structure, and only a small portion of the code implements
the functionality that motivated the traversal in the first place.
This paper present a type-safe Java framework called \name that
removes much of this boilerplate code. In \name \emph{object algebras}
are used to describe complex data structures. Using Java annotations
generic boilerplate code is generated for various types of traversals,
including queries and transformations. Then programmers can inherit
the generic traversal code to focus only on writing the interesting
parts of the traversals. Consequentely, the amount of code that
programmers need to write is significantly smaller, and traversals
using the \name framework are also much more \emph{structure
  shy}. That is, since traversals have less type-specific code they
become much more adaptive to future changes in the data structure.
%An additional benefit of \name is that traversal code can be easily 
%reused by extensions of the 
To prove the effectiveness of the approach, we employeed \name 
on the implementation of a domain-specific questionaire 
language. Our results show that for a large number of traversals 
there was a significant reduction in the amount of user-defined code.
\bruno{Say something more about extensibility and type-safety!}

\begin{comment}

Static types are useful to distinguish between different kinds
of nodes in a structure and to prevent misuses. The
distinction between different types of nodes also means that code for
dealing with each type of node is needed. However, in large structures, 
such as the Abstract Syntax Tree (AST) of a programming language, 
the amount of required code can be a problem. For some operations, 
which traverse large structures, most of the code amounts to recursively 
delegating the traversal to the child nodes. Only for some nodes 
of the structure the code needs to do something different. Still 
the programmer needs to diligently and tediously write the error-prone 
traversal code for all nodes.

In this paper we present a framework 

This approach works well for many operations,
which need different (and non-trivial) code for each different type 
of node 

%Unfortunately, for some operations the interesting 

%%In large structures, 
%%such as the Abstract Syntax Tree (AST) of a programming language, 
%%the code required to traverse the whole structure is proportionaly large.


The problem is particularly prominent 
in statically typed languages, where the typing discipline 
enforces strict distinctions between different cases 

Operations that traverse complex structures often require large and 
tedious amounts of boilerplate code. In those operations there are 
typically a few 

A pervasive problem in programming occurs when large tree tr

\end{comment}

In summary, the contributions of this paper are:

\begin{itemize}

\item {\bf Design patterns for generic traversals.} We provide a set of design 
patterns for various types of traversals using object algebras. These include: 
\emph{queries}, \emph{transformations},
\emph{generalized queries} and \emph{generalized transformations}.

\item {\bf Shy: a Java framework for eliminating boilerplate code.} We have implemented 
a Java framework that can be used to describe complex structures using 
object algebras; and to eliminate boilerplate code. The framework uses Java 
annotations to automatically generate generic traversals.

\item {\bf Case study and empirical evaluation.} We evaluate the approach using 
a case study based on a domain-specific language for questionaires. The results 
of our case study show significant savings in terms of user-defined traversal code.

\end{itemize}
